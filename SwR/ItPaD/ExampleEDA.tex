\documentclass[]{article}
\usepackage{lmodern}
\usepackage{amssymb,amsmath}
\usepackage{ifxetex,ifluatex}
\usepackage{fixltx2e} % provides \textsubscript
\ifnum 0\ifxetex 1\fi\ifluatex 1\fi=0 % if pdftex
  \usepackage[T1]{fontenc}
  \usepackage[utf8]{inputenc}
\else % if luatex or xelatex
  \ifxetex
    \usepackage{mathspec}
  \else
    \usepackage{fontspec}
  \fi
  \defaultfontfeatures{Ligatures=TeX,Scale=MatchLowercase}
\fi
% use upquote if available, for straight quotes in verbatim environments
\IfFileExists{upquote.sty}{\usepackage{upquote}}{}
% use microtype if available
\IfFileExists{microtype.sty}{%
\usepackage{microtype}
\UseMicrotypeSet[protrusion]{basicmath} % disable protrusion for tt fonts
}{}
\usepackage[margin=1in]{geometry}
\usepackage{hyperref}
\hypersetup{unicode=true,
            pdftitle={Example Exploratory Data Analysis},
            pdfborder={0 0 0},
            breaklinks=true}
\urlstyle{same}  % don't use monospace font for urls
\usepackage{color}
\usepackage{fancyvrb}
\newcommand{\VerbBar}{|}
\newcommand{\VERB}{\Verb[commandchars=\\\{\}]}
\DefineVerbatimEnvironment{Highlighting}{Verbatim}{commandchars=\\\{\}}
% Add ',fontsize=\small' for more characters per line
\usepackage{framed}
\definecolor{shadecolor}{RGB}{248,248,248}
\newenvironment{Shaded}{\begin{snugshade}}{\end{snugshade}}
\newcommand{\KeywordTok}[1]{\textcolor[rgb]{0.13,0.29,0.53}{\textbf{#1}}}
\newcommand{\DataTypeTok}[1]{\textcolor[rgb]{0.13,0.29,0.53}{#1}}
\newcommand{\DecValTok}[1]{\textcolor[rgb]{0.00,0.00,0.81}{#1}}
\newcommand{\BaseNTok}[1]{\textcolor[rgb]{0.00,0.00,0.81}{#1}}
\newcommand{\FloatTok}[1]{\textcolor[rgb]{0.00,0.00,0.81}{#1}}
\newcommand{\ConstantTok}[1]{\textcolor[rgb]{0.00,0.00,0.00}{#1}}
\newcommand{\CharTok}[1]{\textcolor[rgb]{0.31,0.60,0.02}{#1}}
\newcommand{\SpecialCharTok}[1]{\textcolor[rgb]{0.00,0.00,0.00}{#1}}
\newcommand{\StringTok}[1]{\textcolor[rgb]{0.31,0.60,0.02}{#1}}
\newcommand{\VerbatimStringTok}[1]{\textcolor[rgb]{0.31,0.60,0.02}{#1}}
\newcommand{\SpecialStringTok}[1]{\textcolor[rgb]{0.31,0.60,0.02}{#1}}
\newcommand{\ImportTok}[1]{#1}
\newcommand{\CommentTok}[1]{\textcolor[rgb]{0.56,0.35,0.01}{\textit{#1}}}
\newcommand{\DocumentationTok}[1]{\textcolor[rgb]{0.56,0.35,0.01}{\textbf{\textit{#1}}}}
\newcommand{\AnnotationTok}[1]{\textcolor[rgb]{0.56,0.35,0.01}{\textbf{\textit{#1}}}}
\newcommand{\CommentVarTok}[1]{\textcolor[rgb]{0.56,0.35,0.01}{\textbf{\textit{#1}}}}
\newcommand{\OtherTok}[1]{\textcolor[rgb]{0.56,0.35,0.01}{#1}}
\newcommand{\FunctionTok}[1]{\textcolor[rgb]{0.00,0.00,0.00}{#1}}
\newcommand{\VariableTok}[1]{\textcolor[rgb]{0.00,0.00,0.00}{#1}}
\newcommand{\ControlFlowTok}[1]{\textcolor[rgb]{0.13,0.29,0.53}{\textbf{#1}}}
\newcommand{\OperatorTok}[1]{\textcolor[rgb]{0.81,0.36,0.00}{\textbf{#1}}}
\newcommand{\BuiltInTok}[1]{#1}
\newcommand{\ExtensionTok}[1]{#1}
\newcommand{\PreprocessorTok}[1]{\textcolor[rgb]{0.56,0.35,0.01}{\textit{#1}}}
\newcommand{\AttributeTok}[1]{\textcolor[rgb]{0.77,0.63,0.00}{#1}}
\newcommand{\RegionMarkerTok}[1]{#1}
\newcommand{\InformationTok}[1]{\textcolor[rgb]{0.56,0.35,0.01}{\textbf{\textit{#1}}}}
\newcommand{\WarningTok}[1]{\textcolor[rgb]{0.56,0.35,0.01}{\textbf{\textit{#1}}}}
\newcommand{\AlertTok}[1]{\textcolor[rgb]{0.94,0.16,0.16}{#1}}
\newcommand{\ErrorTok}[1]{\textcolor[rgb]{0.64,0.00,0.00}{\textbf{#1}}}
\newcommand{\NormalTok}[1]{#1}
\usepackage{longtable,booktabs}
\usepackage{graphicx,grffile}
\makeatletter
\def\maxwidth{\ifdim\Gin@nat@width>\linewidth\linewidth\else\Gin@nat@width\fi}
\def\maxheight{\ifdim\Gin@nat@height>\textheight\textheight\else\Gin@nat@height\fi}
\makeatother
% Scale images if necessary, so that they will not overflow the page
% margins by default, and it is still possible to overwrite the defaults
% using explicit options in \includegraphics[width, height, ...]{}
\setkeys{Gin}{width=\maxwidth,height=\maxheight,keepaspectratio}
\IfFileExists{parskip.sty}{%
\usepackage{parskip}
}{% else
\setlength{\parindent}{0pt}
\setlength{\parskip}{6pt plus 2pt minus 1pt}
}
\setlength{\emergencystretch}{3em}  % prevent overfull lines
\providecommand{\tightlist}{%
  \setlength{\itemsep}{0pt}\setlength{\parskip}{0pt}}
\setcounter{secnumdepth}{0}
% Redefines (sub)paragraphs to behave more like sections
\ifx\paragraph\undefined\else
\let\oldparagraph\paragraph
\renewcommand{\paragraph}[1]{\oldparagraph{#1}\mbox{}}
\fi
\ifx\subparagraph\undefined\else
\let\oldsubparagraph\subparagraph
\renewcommand{\subparagraph}[1]{\oldsubparagraph{#1}\mbox{}}
\fi

%%% Use protect on footnotes to avoid problems with footnotes in titles
\let\rmarkdownfootnote\footnote%
\def\footnote{\protect\rmarkdownfootnote}

%%% Change title format to be more compact
\usepackage{titling}

% Create subtitle command for use in maketitle
\newcommand{\subtitle}[1]{
  \posttitle{
    \begin{center}\large#1\end{center}
    }
}

\setlength{\droptitle}{-2em}

  \title{Example Exploratory Data Analysis}
    \pretitle{\vspace{\droptitle}\centering\huge}
  \posttitle{\par}
    \author{}
    \preauthor{}\postauthor{}
    \date{}
    \predate{}\postdate{}
  

\begin{document}
\maketitle

\hypertarget{instructions}{}
This is an unmarked optional tutorial to show the kind of thinking that
goes into an exploratory data analysis

The goal of this tutorial document is to walk through some of the common
issues encountered in the early stages of an exploratory analysis on a
set of data. It gives examples of common problem areas in:

\begin{itemize}
\tightlist
\item
  reading in data
\item
  dealing with blanks
\item
  dealing with factors
\end{itemize}

This data is a modified version of data from the New Zealand Election
Survey, deliberately modified to introduce problems that occur naturally
in many data sets.

\subsection{Step One. Learn something about the data
set.}\label{step-one.-learn-something-about-the-data-set.}

In this case, the New Zealand Election Survey takes place every three
years as a postal survey of a sample of registered electors. Some
sampled electors were part of a sample panel of people surveyed at the
previous election as part of a longitudinal study, others were randomly
chosen from the electoral roll. Those electors that were part of the
longitudinal panel group were randomly selected in previous elections.

As well as survey results, the data set includes information from the
electoral roll, and weighting values for adjusting results. The full
NZES data set has been reduced to a selected group of variables, making
3101 observations of 107 variables.

\subsection{Step Two. Contemplate some
questions.}\label{step-two.-contemplate-some-questions.}

Examining the codebook (or in this case the appendix at the end of the
document to check out the variables).

For example, we might decide that since New Zealand is a Mixed Member
Proportional voting system, where people get to vote for both an
electorate (local) representative and a nationwide political party, that
it would be interesting to look at strategic voting under conditions
where there are many political parties to choose from. We identify some
relevant variables of interest in the data, and investigate the nature
of the individual variables before we explore their interactions. The
kind of variables they are is going to shape our question.

\subsubsection{Read in the data}\label{read-in-the-data}

There are many different kinds of data files in the world. Each one has
its own issues when being read in by R. In this case the data is saved
as a .RData file, which can be read in by using the \texttt{load()}
command.

As with most reading in file commands, inside the parentheses needs to
go a piece of text, in quotes, that is the path to the file from the
working directory (the working directory is the folder that R is
currently paying attention to). The easiest way to get a R Markdown
(Rmd) Document and console cooperating about this is to place the file
with the data in it in the same folder as the R Markdown (Rmd) Document,
open the R Markdown (Rmd) document in RStudio so we are looking at the
contents of the document in the editing window, then in the RStudio
Session menu, use the \textbf{Set Working Directory - To Source File
Location} command to make a common starting point. Then the code in the
R Markdown (Rmd) document will use the same working space regardless of
whether we knit the document or run code chunks in the Console. In this
case, if the \texttt{nzes2011.RData} file is in the same folder as the R
Markdown (Rmd) document, hence it can be read into R with the following
command:

\begin{Shaded}
\begin{Highlighting}[]
\KeywordTok{load}\NormalTok{(}\StringTok{"selected_nzes2011.Rdata"}\NormalTok{)}
\end{Highlighting}
\end{Shaded}

We also want to load packages that have functions in them we want to
use. For this particular analysis we will only need the \texttt{dplyr}
package, but for your project you will also likely need other packages
as well, e.g. \texttt{ggplot2}.

\begin{Shaded}
\begin{Highlighting}[]
\KeywordTok{library}\NormalTok{(dplyr)}
\end{Highlighting}
\end{Shaded}

\begin{verbatim}
## Warning: package 'dplyr' was built under R version 3.5.2
\end{verbatim}

\subsection{Step Three. Prepare for the first
question}\label{step-three.-prepare-for-the-first-question}

As a first question, we might be interested in exploring the
relationship between the party the person voted for, the party that was
their favourite, and if they believed that their vote makes a difference
-- focusing on the question that are people who believe their vote makes
a difference more likely to strategically vote for a party not their
favourite. To achieve this, we familiarise ourselves with the variables
\texttt{jpartyvote}, \texttt{jdiffvoting}, and \texttt{\_singlefav}.
First we check the codebook (see Appendix), then we explore the data.

Viewing the entire dataset in the Data Viewer window by clicking on the
data frame's name in the Environment or running the \texttt{View()}
command in the Console can be ineffective since the Data Viewer only
shows the first 100 columns of the data frame.

Using the \texttt{str()} command on the entire dataset can also be
equally ineffective. However we can subset the columns of interest and
take a closer look at them. We can use the \texttt{dplyr} chain to
select the variables of interest and investigate only their structure by
adding \texttt{str()} at the end of the chain:

\begin{Shaded}
\begin{Highlighting}[]
\NormalTok{selected_nzes2011 }\OperatorTok\StringTok{ }
\StringTok{  }\KeywordTok{select}\NormalTok{(jpartyvote, jdiffvoting, _singlefav) }\OperatorTok\StringTok{ }
\StringTok{  }\KeywordTok{str}\NormalTok{()}
\end{Highlighting}
\end{Shaded}

If we try to run that line, we will get an error message about
unexpected input or missing object.

We next need to diagnose where the problem lies -- in the R code or in
the data? The best way to troubleshoot this issue is to run each line of
the \texttt{dplyr} chain one by one.

\begin{Shaded}
\begin{Highlighting}[]
\NormalTok{selected_nzes2011 }
\end{Highlighting}
\end{Shaded}

The first line runs without any erros, but the second line gives an
error

\begin{Shaded}
\begin{Highlighting}[]
\NormalTok{selected_nzes2011 }\OperatorTok\StringTok{ }
\StringTok{  }\KeywordTok{select}\NormalTok{(jpartyvote, jdiffvoting, _singlefav)}
\end{Highlighting}
\end{Shaded}

We know that \texttt{select()} is a valid \texttt{dplyr} function, so
that cannot be the problem. This means the problem might be the variable
names. The issue is that R has rules about what variable names are legal
(e.g.~no spaces, starting with a letter) and when data is loaded, R will
often fix variable names to make them legal. This happened to the
\texttt{\_singlefav} at the time of loading the data.

We could check this by looking through every single variable name in the
data with the \texttt{names()} command.

\begin{Shaded}
\begin{Highlighting}[]
\KeywordTok{names}\NormalTok{(selected_nzes2011)}
\end{Highlighting}
\end{Shaded}

\begin{verbatim}
##   [1] "Jelect"         "jblogel"        "jnewspaper"     "jnatradio"     
##   [5] "jtalkback"      "jdiscussp"      "jrallies"       "jpersuade"     
##   [9] "jpcmoney"       "jpcposter"      "jlablike"       "jnatlike"      
##  [13] "jgrnlike"       "jnzflike"       "jactlike"       "junflike"      
##  [17] "jmaolike"       "jmnplike"       "jmostlike"      "jmostlikex"    
##  [21] "jrepublic"      "jsphealth"      "jspedu"         "jspunemp"      
##  [25] "jspdefence"     "jspsuper"       "jspbusind"      "jsppolice"     
##  [29] "jspwelfare"     "jspenviro"      "jgovpdk"        "jgovplab"      
##  [33] "jgovpnat"       "jgovpgrn"       "jgovpnzf"       "jgovpact"      
##  [37] "jgovunf"        "jgovpmao"       "jgovpmnp"       "jnevervoteno"  
##  [41] "jnevervotelab"  "jnevervotenat"  "jnevervotegrn"  "jnevervotenzf" 
##  [45] "jnevervoteact"  "jnevervoteunf"  "jnevervotemao"  "jnevervotemnp" 
##  [49] "jnevervoteoth"  "jnevervoteothx" "jfirstpx"       "jsecondp"      
##  [53] "jage"           "jlanguage"      "jlanguagex"     "jrollsex"      
##  [57] "jhqual"         "jwkft"          "jwkpt"          "jwkun"         
##  [61] "jwkret"         "jwkdis"         "jwksch"         "jwkunpo"       
##  [65] "jwkunpi"        "jhhincome"      "jhhadults"      "jhhchn"        
##  [69] "jmarital"       "r_jind"         "jlablr"         "jnatlr"        
##  [73] "jgrnlr"         "jnzflr"         "jactlr"         "junflr"        
##  [77] "jmaolr"         "jmnplr"         "jslflr"         "jrelservices"  
##  [81] "jrelnone"       "jrelang"        "jrelpres"       "jrelcath"      
##  [85] "jrelmeth"       "jrelbap"        "jrellat"        "jrelrat"       
##  [89] "jrelfun"        "jrelothc"       "jrelnonc"       "jreligionx"    
##  [93] "jreligiousity"  "jethnicity_e"   "jethnicity_m"   "jethnicity_p"  
##  [97] "jethnicity_a"   "jethnicity_o"   "jethnicityx"    "jethnicmost"   
## [101] "jethnicmostx"   "jpartyvote"     "jelecvote"      "njptyvote"     
## [105] "njelecvote"     "jdiffvoting"    "X_singlefav"
\end{verbatim}

However, when we have hundreds of column names, a useful tip is to just
search out only possible names. We can search the names for a fragment
of the name by using the
\texttt{grep("FRAGMENT",\ variable,\ value\ =\ TRUE)} command, which in
this case might be:

\begin{Shaded}
\begin{Highlighting}[]
\KeywordTok{grep}\NormalTok{(}\StringTok{"singlefav"}\NormalTok{, }\KeywordTok{names}\NormalTok{(selected_nzes2011), }\DataTypeTok{value =} \OtherTok{TRUE}\NormalTok{)}
\end{Highlighting}
\end{Shaded}

\begin{verbatim}
## [1] "X_singlefav"
\end{verbatim}

The \texttt{value\ =\ TRUE} argument, as described in the help for the
\texttt{grep()} function reports the mathing character string, as
opposed to the index number for that string.

We can now confirm that the variable is called \texttt{X\_singlefav}, so
that is how we should be referring to it.

\begin{Shaded}
\begin{Highlighting}[]
\NormalTok{selected_nzes2011 }\OperatorTok\StringTok{ }
\StringTok{  }\KeywordTok{select}\NormalTok{(jpartyvote, jdiffvoting, X_singlefav) }\OperatorTok\StringTok{ }
\StringTok{  }\KeywordTok{str}\NormalTok{()}
\end{Highlighting}
\end{Shaded}

These are all categorical data, however they are recorded as characters
(text strings) as opposed to factors.

An easy way of tabulating these data to see how many times each level of
is to use the \texttt{group\_by()} function along with the
\texttt{summarise()} command:

\begin{Shaded}
\begin{Highlighting}[]
\NormalTok{selected_nzes2011 }\OperatorTok\StringTok{ }
\StringTok{  }\KeywordTok{group_by}\NormalTok{(jpartyvote) }\OperatorTok\StringTok{ }
\StringTok{  }\KeywordTok{summarise}\NormalTok{(}\DataTypeTok{count =} \KeywordTok{n}\NormalTok{())}
\end{Highlighting}
\end{Shaded}

\begin{verbatim}
## # A tibble: 14 x 2
##    jpartyvote    count
##    <chr>         <int>
##  1 <NA>            308
##  2 Act              29
##  3 ALC              10
##  4 Alliance          2
##  5 Another party     8
##  6 Conservative     74
##  7 Don't know       23
##  8 Green           348
##  9 Labour          749
## 10 Mana             62
## 11 Maori Party     128
## 12 National       1130
## 13 NZ First        216
## 14 United Future    14
\end{verbatim}

We can see that 23 people answered
\texttt{"Don\textquotesingle{}t\ know"}. Since our question is about
people who knew which party they voted for, we might want to exclude
these observations from our analysis. We can do so by \texttt{filter}ing
them out.

\begin{Shaded}
\begin{Highlighting}[]
\NormalTok{selected_nzes2011 }\OperatorTok\StringTok{ }
\StringTok{  }\KeywordTok{filter}\NormalTok{(jpartyvote }\OperatorTok{!=}\StringTok{ "Don't know"}\NormalTok{) }\OperatorTok
\StringTok{  }\KeywordTok{group_by}\NormalTok{(jpartyvote) }\OperatorTok\StringTok{ }
\StringTok{  }\KeywordTok{summarise}\NormalTok{(}\DataTypeTok{count =} \KeywordTok{n}\NormalTok{())}
\end{Highlighting}
\end{Shaded}

\begin{verbatim}
## # A tibble: 12 x 2
##    jpartyvote    count
##    <chr>         <int>
##  1 Act              29
##  2 ALC              10
##  3 Alliance          2
##  4 Another party     8
##  5 Conservative     74
##  6 Green           348
##  7 Labour          749
##  8 Mana             62
##  9 Maori Party     128
## 10 National       1130
## 11 NZ First        216
## 12 United Future    14
\end{verbatim}

Because there is a \texttt{\%\textgreater{}\%} at the end of the line, R
knows to continue on to the next line, as with any other `to be
continued' symbol at the end of the line.

Note that adding the filter also got rid of the \texttt{NA} entries. NA
(Not Available) is used to indicate blank entries -- those observations
for which there is no data recorded. It is always a good plan to be
aware of NAs and deliberately include them in or exclude them from the
analysis so that the final results are not surprising. In this case
since NA indicates that these people did not answer the question about
which party they voted for, exluding them from the analysis makes sense.

We can also similarly view the levels and number of occurances of these
levels in the \texttt{X\_singlefav} variable:

\begin{Shaded}
\begin{Highlighting}[]
\NormalTok{selected_nzes2011 }\OperatorTok\StringTok{ }
\StringTok{  }\KeywordTok{group_by}\NormalTok{(X_singlefav) }\OperatorTok\StringTok{ }
\StringTok{  }\KeywordTok{summarise}\NormalTok{(}\DataTypeTok{count =} \KeywordTok{n}\NormalTok{())}
\end{Highlighting}
\end{Shaded}

\begin{verbatim}
## # A tibble: 8 x 2
##   X_singlefav   count
##   <chr>         <int>
## 1 <NA>             58
## 2 Act              33
## 3 Green           388
## 4 Labour         1043
## 5 Mana             47
## 6 National       1266
## 7 NZ First        138
## 8 United Future   128
\end{verbatim}

This set also has \texttt{NA} entries, but in this case we don't want to
get rid of anything but the \texttt{NA}s so we need to target them
directly. \texttt{NA} entries need special targeting because they do not
actually exist (they are different to the text \texttt{"NA"} or a
variable saved with the name \texttt{NA}).

If we only wanted to find the \texttt{NA}s we would use the
\texttt{is.na()} function with the name of the variable inside the
parentheses.

However since we want the entries that are \textbf{not} \texttt{NA}s we
can use the \textbf{Not} operator, \texttt{!}, to indicate ``we want all
the ones that are not NA'':\texttt{!is.na()}. Hence we can
\texttt{filter} out all non NAs in our \texttt{dplyr} chain:

\begin{Shaded}
\begin{Highlighting}[]
\NormalTok{selected_nzes2011 }\OperatorTok\StringTok{ }
\StringTok{  }\KeywordTok{filter}\NormalTok{(}\OperatorTok{!}\KeywordTok{is.na}\NormalTok{(X_singlefav)) }\OperatorTok
\StringTok{  }\KeywordTok{group_by}\NormalTok{(X_singlefav) }\OperatorTok\StringTok{ }
\StringTok{  }\KeywordTok{summarise}\NormalTok{(}\DataTypeTok{count =} \KeywordTok{n}\NormalTok{())}
\end{Highlighting}
\end{Shaded}

\begin{verbatim}
## # A tibble: 7 x 2
##   X_singlefav   count
##   <chr>         <int>
## 1 Act              33
## 2 Green           388
## 3 Labour         1043
## 4 Mana             47
## 5 National       1266
## 6 NZ First        138
## 7 United Future   128
\end{verbatim}

And remember that we can \texttt{filter} for multiple characteristics at
once:

\begin{Shaded}
\begin{Highlighting}[]
\NormalTok{selected_nzes2011 }\OperatorTok\StringTok{ }
\StringTok{  }\KeywordTok{filter}\NormalTok{(}\OperatorTok{!}\KeywordTok{is.na}\NormalTok{(X_singlefav), jpartyvote }\OperatorTok{!=}\StringTok{ "Don't know"}\NormalTok{) }\OperatorTok
\StringTok{  }\KeywordTok{group_by}\NormalTok{(X_singlefav) }\OperatorTok\StringTok{ }
\StringTok{  }\KeywordTok{summarise}\NormalTok{(}\DataTypeTok{count=}\KeywordTok{n}\NormalTok{())}
\end{Highlighting}
\end{Shaded}

\begin{verbatim}
## # A tibble: 7 x 2
##   X_singlefav   count
##   <chr>         <int>
## 1 Act              29
## 2 Green           354
## 3 Labour          914
## 4 Mana             42
## 5 National       1172
## 6 NZ First        119
## 7 United Future   115
\end{verbatim}

If we examine the categories in \texttt{jdiffvoting} we can see that
this variable has levels such as both
\texttt{"Don\textquotesingle{}t\ know"} and \texttt{NA}.

\begin{Shaded}
\begin{Highlighting}[]
\NormalTok{selected_nzes2011 }\OperatorTok\StringTok{ }
\StringTok{  }\KeywordTok{group_by}\NormalTok{(jdiffvoting) }\OperatorTok\StringTok{ }
\StringTok{  }\KeywordTok{summarise}\NormalTok{(}\DataTypeTok{count =} \KeywordTok{n}\NormalTok{())}
\end{Highlighting}
\end{Shaded}

\begin{verbatim}
## # A tibble: 7 x 2
##   jdiffvoting                                                       count
##   <chr>                                                             <int>
## 1 <NA>                                                                 28
## 2 Don't know                                                           63
## 3 Voting can make a big difference to what happens                   1605
## 4 Voting can make a reasonable amount of difference to what happens   841
## 5 Voting can make some difference to what happens                     339
## 6 Voting won't make any difference to what happens                    119
## 7 Voting won't make much difference to what happens                   106
\end{verbatim}

We need to decide how we want to handle these levels in our analysis.

Remember that our main question is about whether people vote for their
favorite party or a diffent one. Hence an straighforwrd approach would
be to first determine whether each observation in the data represents a
person who voted for the party same as their favorite party or
different. This requires creating a new variable with the
\texttt{mutate()} function.

In creating this variable we want to evaluate if for a given observation
the values in the \texttt{jpartyvote} and \texttt{X\_singlefav}
variables are the same, or different:

\begin{Shaded}
\begin{Highlighting}[]
\NormalTok{selected_nzes2011 <-}\StringTok{ }\NormalTok{selected_nzes2011 }\OperatorTok
\StringTok{  }\KeywordTok{mutate}\NormalTok{(}\DataTypeTok{sameparty =} \KeywordTok{ifelse}\NormalTok{(jpartyvote }\OperatorTok{==}\StringTok{ }\NormalTok{X_singlefav, }\StringTok{"same"}\NormalTok{, }\StringTok{"different"}\NormalTok{))}
\end{Highlighting}
\end{Shaded}

This creates a new variable named \texttt{sameparty} that has the value
\texttt{"same"} if \texttt{jpartyvote} is equal to
\texttt{X\_singlefav}, and \texttt{"different"} otherwise.

We can again check our work by exploring the groupings in a View:

\begin{Shaded}
\begin{Highlighting}[]
\NormalTok{selected_nzes2011 }\OperatorTok\StringTok{ }
\KeywordTok{group_by}\NormalTok{(jpartyvote, X_singlefav, sameparty) }\OperatorTok
\StringTok{  }\KeywordTok{summarise}\NormalTok{(}\DataTypeTok{count =} \KeywordTok{n}\NormalTok{())}
\end{Highlighting}
\end{Shaded}

\begin{verbatim}
## # A tibble: 82 x 4
## # Groups:   jpartyvote, X_singlefav [82]
##    jpartyvote X_singlefav   sameparty count
##    <chr>      <chr>         <chr>     <int>
##  1 <NA>       <NA>          <NA>         26
##  2 <NA>       Act           <NA>          4
##  3 <NA>       Green         <NA>         32
##  4 <NA>       Labour        <NA>        121
##  5 <NA>       Mana          <NA>          4
##  6 <NA>       National      <NA>         92
##  7 <NA>       NZ First      <NA>         17
##  8 <NA>       United Future <NA>         12
##  9 Act        <NA>          <NA>          1
## 10 Act        Act           same         12
## # ... with 72 more rows
\end{verbatim}

We can see that observations where \texttt{jpartyvote} equaled
\texttt{X\_singlefav}, the value \texttt{"same"} was recorded for the
new variable \texttt{sameparty}, and the value \texttt{"different"} was
recorded otherwise. If either \texttt{jpartyvote} or
\texttt{X\_singlefav} had an \texttt{NA}, R could not check for equality
and hence \texttt{NA} was recorded for the \texttt{sameparty} variable
as well.

To view and summarize the ``same'' entries we can use the following:

\begin{Shaded}
\begin{Highlighting}[]
\NormalTok{selected_nzes2011 }\OperatorTok\StringTok{ }
\StringTok{  }\KeywordTok{group_by}\NormalTok{(jpartyvote, X_singlefav, sameparty) }\OperatorTok
\StringTok{  }\KeywordTok{summarise}\NormalTok{(}\DataTypeTok{count =} \KeywordTok{n}\NormalTok{()) }\OperatorTok\StringTok{ }
\StringTok{  }\KeywordTok{filter}\NormalTok{(sameparty }\OperatorTok{==}\StringTok{ "same"}\NormalTok{)}
\end{Highlighting}
\end{Shaded}

\begin{verbatim}
## # A tibble: 7 x 4
## # Groups:   jpartyvote, X_singlefav [7]
##   jpartyvote    X_singlefav   sameparty count
##   <chr>         <chr>         <chr>     <int>
## 1 Act           Act           same         12
## 2 Green         Green         same        237
## 3 Labour        Labour        same        632
## 4 Mana          Mana          same         31
## 5 National      National      same       1004
## 6 NZ First      NZ First      same         82
## 7 United Future United Future same          5
\end{verbatim}

And to view and summarize the ``different'' entries we can use the
following:

\begin{Shaded}
\begin{Highlighting}[]
\NormalTok{selected_nzes2011 }\OperatorTok\StringTok{ }
\StringTok{  }\KeywordTok{group_by}\NormalTok{(jpartyvote, X_singlefav, sameparty) }\OperatorTok
\StringTok{  }\KeywordTok{summarise}\NormalTok{(}\DataTypeTok{count =} \KeywordTok{n}\NormalTok{()) }\OperatorTok\StringTok{ }
\StringTok{  }\KeywordTok{filter}\NormalTok{(sameparty }\OperatorTok{==}\StringTok{ "different"}\NormalTok{)}
\end{Highlighting}
\end{Shaded}

\begin{verbatim}
## # A tibble: 59 x 4
## # Groups:   jpartyvote, X_singlefav [59]
##    jpartyvote    X_singlefav   sameparty count
##    <chr>         <chr>         <chr>     <int>
##  1 Act           Green         different     1
##  2 Act           National      different    14
##  3 Act           United Future different     1
##  4 ALC           Green         different     1
##  5 ALC           Labour        different     4
##  6 ALC           National      different     2
##  7 ALC           United Future different     3
##  8 Alliance      Labour        different     1
##  9 Alliance      National      different     1
## 10 Another party Green         different     2
## # ... with 49 more rows
\end{verbatim}

We can also check how we got any \texttt{NA}s we have by using the
\texttt{is.na()} function:

\begin{Shaded}
\begin{Highlighting}[]
\NormalTok{selected_nzes2011 }\OperatorTok\StringTok{ }
\StringTok{  }\KeywordTok{group_by}\NormalTok{(jpartyvote, X_singlefav, sameparty) }\OperatorTok
\StringTok{  }\KeywordTok{summarise}\NormalTok{(}\DataTypeTok{count =} \KeywordTok{n}\NormalTok{()) }\OperatorTok\StringTok{ }
\StringTok{  }\KeywordTok{filter}\NormalTok{(}\KeywordTok{is.na}\NormalTok{(sameparty))}
\end{Highlighting}
\end{Shaded}

\begin{verbatim}
## # A tibble: 16 x 4
## # Groups:   jpartyvote, X_singlefav [16]
##    jpartyvote   X_singlefav   sameparty count
##    <chr>        <chr>         <chr>     <int>
##  1 <NA>         <NA>          <NA>         26
##  2 <NA>         Act           <NA>          4
##  3 <NA>         Green         <NA>         32
##  4 <NA>         Labour        <NA>        121
##  5 <NA>         Mana          <NA>          4
##  6 <NA>         National      <NA>         92
##  7 <NA>         NZ First      <NA>         17
##  8 <NA>         United Future <NA>         12
##  9 Act          <NA>          <NA>          1
## 10 Conservative <NA>          <NA>          1
## 11 Don't know   <NA>          <NA>          7
## 12 Green        <NA>          <NA>          1
## 13 Labour       <NA>          <NA>         11
## 14 Maori Party  <NA>          <NA>          2
## 15 National     <NA>          <NA>          7
## 16 NZ First     <NA>          <NA>          2
\end{verbatim}

The checks show that the observations with \texttt{NA}s in the
\texttt{sameparty}are going to be excluded from the analysis when we
fiter out the \texttt{NA}s in the \texttt{jpartyvote} and
\texttt{X\_singlefav} variables, so we don't need to worry about them
anymore.

\subsection{Step four. Prepare for the second
question}\label{step-four.-prepare-for-the-second-question}

As a second question, we might be interested in exploring the
relationship between age of voters and how much they like the NZ First
party. We become familiar with the variables \texttt{jnzflike} and
\texttt{jage} in the codebook, then explore the data.

\begin{Shaded}
\begin{Highlighting}[]
\KeywordTok{str}\NormalTok{(selected_nzes2011}\OperatorTok{$}\NormalTok{jnzflike)}
\end{Highlighting}
\end{Shaded}

\begin{verbatim}
##  Factor w/ 12 levels "0","1","10","2",..: 1 1 4 10 4 11 NA NA 1 12 ...
\end{verbatim}

\begin{Shaded}
\begin{Highlighting}[]
\KeywordTok{str}\NormalTok{(selected_nzes2011}\OperatorTok{$}\NormalTok{jage)}
\end{Highlighting}
\end{Shaded}

\begin{verbatim}
##  int [1:3101] 37 37 28 71 43 NA 59 68 64 70 ...
\end{verbatim}

\texttt{jnzflike} is a factor variable, in fact it's ordinal and by
default the levels are listed in alphabetical order. Since this is a
categorical variable, we can also summarize the occurances of each level
with \texttt{group\_by()} and \texttt{summarise()} again:

\begin{Shaded}
\begin{Highlighting}[]
\NormalTok{selected_nzes2011 }\OperatorTok\StringTok{ }
\StringTok{  }\KeywordTok{group_by}\NormalTok{(jnzflike) }\OperatorTok\StringTok{ }
\StringTok{  }\KeywordTok{summarise}\NormalTok{(}\DataTypeTok{count =} \KeywordTok{n}\NormalTok{())}
\end{Highlighting}
\end{Shaded}

\begin{verbatim}
## Warning: Factor `jnzflike` contains implicit NA, consider using
## `forcats::fct_explicit_na`
\end{verbatim}

\begin{verbatim}
## # A tibble: 13 x 2
##    jnzflike   count
##    <fct>      <int>
##  1 0            622
##  2 1            298
##  3 10           134
##  4 2            266
##  5 3            227
##  6 4            162
##  7 5            544
##  8 6            165
##  9 7            138
## 10 8            107
## 11 9             81
## 12 Don't know   224
## 13 <NA>         133
\end{verbatim}

While \texttt{jnzflike} is on a 0 to 10 scale, this variable also has a
level labeled \texttt{"Don\textquotesingle{}t\ know"}, which is why R
stores this variable as not a numeric variable.

\texttt{jage}, on the other hand, is an integer, with values that are
whole numbers between 0 and infinity (or \texttt{NA}). For this variable
we would want to take a look at numerical summaries such as means,
medians, etc.

\begin{Shaded}
\begin{Highlighting}[]
\NormalTok{selected_nzes2011 }\OperatorTok\StringTok{ }
\StringTok{  }\KeywordTok{summarise}\NormalTok{(}\DataTypeTok{agemean =} \KeywordTok{mean}\NormalTok{(jage), }\DataTypeTok{agemedian =} \KeywordTok{median}\NormalTok{(jage), }\DataTypeTok{agesd =} \KeywordTok{sd}\NormalTok{(jage), }
            \DataTypeTok{agemin =} \KeywordTok{min}\NormalTok{(jage), }\DataTypeTok{agemax =} \KeywordTok{max}\NormalTok{(jage))}
\end{Highlighting}
\end{Shaded}

\begin{verbatim}
##   agemean agemedian agesd agemin agemax
## 1      NA        NA   NaN     NA     NA
\end{verbatim}

What went wrong? The reason why all of the results were reported as NAs
is that there were some NA entries in the \texttt{jage} variable (people
not reporting their age). Since it is not possible to take the average
of a series of values that contain \texttt{NA}s, obtaining the numerical
summaries requires that we exclude the \texttt{NA}s from the
calculation.

Most numerical summary functions allow us to easily exclude \texttt{NA}s
with the \texttt{na.rm} argument. See the help documentation for the
\texttt{median} function for more information.

\begin{Shaded}
\begin{Highlighting}[]
\NormalTok{?median}
\end{Highlighting}
\end{Shaded}

\begin{verbatim}
## starting httpd help server ... done
\end{verbatim}

An alternative approach is just to \texttt{filter} out the \texttt{NA}s
first, and then ask for the numerical summaries:

\begin{Shaded}
\begin{Highlighting}[]
\NormalTok{selected_nzes2011 }\OperatorTok\StringTok{ }
\StringTok{  }\KeywordTok{filter}\NormalTok{(}\OperatorTok{!}\NormalTok{(}\KeywordTok{is.na}\NormalTok{(jage))) }\OperatorTok
\StringTok{  }\KeywordTok{summarise}\NormalTok{(}\DataTypeTok{agemean =} \KeywordTok{mean}\NormalTok{(jage), }\DataTypeTok{agemedian =} \KeywordTok{median}\NormalTok{(jage), }\DataTypeTok{agesd =} \KeywordTok{sd}\NormalTok{(jage), }
            \DataTypeTok{agemin =} \KeywordTok{min}\NormalTok{(jage), }\DataTypeTok{agemax =} \KeywordTok{max}\NormalTok{(jage))}
\end{Highlighting}
\end{Shaded}

\begin{verbatim}
##    agemean agemedian   agesd agemin agemax
## 1 53.22328        54 17.5371     18    100
\end{verbatim}

An age range of 18 to 100 is a reasonable age range for a voting age
population, so there are no obvious errors in the data. If there were,
we would need to decide if we should filter them out of the analysis.

Having gained some familiarity with the specific variables we are using,
we next need to consider if there is additional work we should do on the
data in investigating the question. There are a number of different
approaches we might take. For example, we could consider if those that
strongly like NZ First are older than those that strongly dislike NZ
First, or we could consider if old people like NZ First more than young
people.

\subsubsection{Approach 1: Strongly liking and disliking NZ First and
age}\label{approach-1-strongly-liking-and-disliking-nz-first-and-age}

If we wanted to select only two of the possible levels in how much
people like NZ First, we can filter for these specific levels. When
interested in filtering for multiple values a variable can take, the
\texttt{\%in\%} operator can come in handy:

\begin{Shaded}
\begin{Highlighting}[]
\NormalTok{selected_nzes2011 }\OperatorTok\StringTok{ }
\StringTok{  }\KeywordTok{filter}\NormalTok{(jnzflike }\OperatorTok\StringTok{ }\KeywordTok{c}\NormalTok{(}\StringTok{"0"}\NormalTok{,}\StringTok{"10"}\NormalTok{)) }\OperatorTok
\StringTok{  }\KeywordTok{group_by}\NormalTok{(jnzflike) }\OperatorTok\StringTok{ }
\StringTok{  }\KeywordTok{summarise}\NormalTok{(}\DataTypeTok{count =} \KeywordTok{n}\NormalTok{())}
\end{Highlighting}
\end{Shaded}

\begin{verbatim}
## # A tibble: 2 x 2
##   jnzflike count
##   <fct>    <int>
## 1 0          622
## 2 10         134
\end{verbatim}

Remember that the \texttt{jnzflike} is not a numerical variable, hence
we use the quotation marks around the values (even though they happen to
be numbers).

This is an example of simpligying the analysis by considering only two
levels of a categorical variable, as opposed to all possible levels.

\subsubsection{Approach 2: Age and liking for NZ
First}\label{approach-2-age-and-liking-for-nz-first}

We might also like to refine our question slightly, asking do people
above retirement age (65 in New Zealand) like NZ First more than younger
people. To do this we can turn the numeric age variable into a
categorical variable based on whether people are 65 years or older or
younger than 65. Once again we make use of the \texttt{mutate()} and
\texttt{ifelse()} functions:

\begin{Shaded}
\begin{Highlighting}[]
\NormalTok{selected_nzes2011 <-}\StringTok{ }\NormalTok{selected_nzes2011 }\OperatorTok\StringTok{ }
\StringTok{  }\KeywordTok{mutate}\NormalTok{(}\DataTypeTok{retiredage =} \KeywordTok{ifelse}\NormalTok{(jage }\OperatorTok{>=}\StringTok{ }\DecValTok{65}\NormalTok{, }\StringTok{"retired age"}\NormalTok{, }\StringTok{"working age"}\NormalTok{))}
\NormalTok{selected_nzes2011 }\OperatorTok\StringTok{ }
\StringTok{  }\KeywordTok{group_by}\NormalTok{(retiredage) }\OperatorTok\StringTok{ }
\StringTok{  }\KeywordTok{summarise}\NormalTok{(}\DataTypeTok{count =} \KeywordTok{n}\NormalTok{())}
\end{Highlighting}
\end{Shaded}

\begin{verbatim}
## # A tibble: 3 x 2
##   retiredage  count
##   <chr>       <int>
## 1 <NA>           69
## 2 retired age   876
## 3 working age  2156
\end{verbatim}

We can see that individuals in the dataset are now labeled as either
\texttt{"retired\ age"} or \texttt{"working\ age"} or neither
(\texttt{NA}), which we can easily filter out if need be.

This is an example of using a numerical threshold to convert a numerical
variable to a categorical variable.

For approach 2, we might also be want to turn the scale of liking into
numeric values, because at the moment we cannot easily get summary
information of the data in factor form. For example, if we ty to run the
following command, we get an error saying ``need numeric data''.

\begin{Shaded}
\begin{Highlighting}[]
\NormalTok{selected_nzes2011 }\OperatorTok\StringTok{ }
\StringTok{  }\KeywordTok{group_by}\NormalTok{(retiredage) }\OperatorTok\StringTok{ }
\StringTok{  }\KeywordTok{summarise}\NormalTok{(}\DataTypeTok{medlike =} \KeywordTok{median}\NormalTok{(jnzflike))}
\end{Highlighting}
\end{Shaded}

it generates a ``need numeric data'' error.

We can change the type of data with functions of the form
\texttt{as.thingtochangeto()}, but it is easy to go wrong with factors.
For example, this is wrong:

\begin{Shaded}
\begin{Highlighting}[]
\NormalTok{selected_nzes2011 <-}\StringTok{ }\NormalTok{selected_nzes2011 }\OperatorTok\StringTok{ }
\StringTok{  }\KeywordTok{mutate}\NormalTok{(}\DataTypeTok{numlikenzf =} \KeywordTok{as.numeric}\NormalTok{(jnzflike))}
\end{Highlighting}
\end{Shaded}

We can see it has gone wrong if we use grouping to check our work (and
it is a very good plan to check our work after converting factors).

\begin{Shaded}
\begin{Highlighting}[]
\NormalTok{selected_nzes2011 }\OperatorTok\StringTok{ }
\StringTok{  }\KeywordTok{group_by}\NormalTok{(jnzflike, numlikenzf) }\OperatorTok\StringTok{ }
\StringTok{  }\KeywordTok{summarise}\NormalTok{(}\DataTypeTok{count =} \KeywordTok{n}\NormalTok{())}
\end{Highlighting}
\end{Shaded}

\begin{verbatim}
## Warning: Factor `jnzflike` contains implicit NA, consider using
## `forcats::fct_explicit_na`

## Warning: Factor `jnzflike` contains implicit NA, consider using
## `forcats::fct_explicit_na`
\end{verbatim}

\begin{verbatim}
## # A tibble: 13 x 3
## # Groups:   jnzflike [13]
##    jnzflike   numlikenzf count
##    <fct>           <dbl> <int>
##  1 0                   1   622
##  2 1                   2   298
##  3 10                  3   134
##  4 2                   4   266
##  5 3                   5   227
##  6 4                   6   162
##  7 5                   7   544
##  8 6                   8   165
##  9 7                   9   138
## 10 8                  10   107
## 11 9                  11    81
## 12 Don't know         12   224
## 13 <NA>               NA   133
\end{verbatim}

Factor entries have two parts: the text we see on the screen, and a
numeric order (remember how 10 was coming between 1 and 2 because of the
alphabetical order). When we say ``turn this into a number'', R uses the
numeric order in which it stores the values to do that conversion, as
opposed to the names of the levels of the categorical variable. Hence,
we need a conversion method that will use the text strings that label
the levels, as opposed to the storage order of these levels. We can do
this by first saving the variable as a character variable, and then
turning it into a number:

\begin{Shaded}
\begin{Highlighting}[]
\NormalTok{selected_nzes2011 <-}\StringTok{ }\NormalTok{selected_nzes2011 }\OperatorTok\StringTok{ }
\StringTok{  }\KeywordTok{mutate}\NormalTok{(}\DataTypeTok{numlikenzf =} \KeywordTok{as.numeric}\NormalTok{(}\KeywordTok{as.character}\NormalTok{(jnzflike)))}
\end{Highlighting}
\end{Shaded}

\begin{verbatim}
## Warning: NAs durch Umwandlung erzeugt
\end{verbatim}

The warning ``NAs introduced by coercion'' happens since the level
\texttt{"Don\textquotesingle{}t\ know"} cannot be turned into a number.
But this should be fine for our purposes since we are interested in the
numerical responses anyway.

\begin{Shaded}
\begin{Highlighting}[]
\NormalTok{selected_nzes2011 }\OperatorTok\StringTok{ }
\StringTok{  }\KeywordTok{group_by}\NormalTok{(jnzflike, numlikenzf) }\OperatorTok\StringTok{ }
\StringTok{  }\KeywordTok{summarise}\NormalTok{(}\DataTypeTok{count =} \KeywordTok{n}\NormalTok{())}
\end{Highlighting}
\end{Shaded}

\begin{verbatim}
## Warning: Factor `jnzflike` contains implicit NA, consider using
## `forcats::fct_explicit_na`

## Warning: Factor `jnzflike` contains implicit NA, consider using
## `forcats::fct_explicit_na`
\end{verbatim}

\begin{verbatim}
## # A tibble: 13 x 3
## # Groups:   jnzflike [13]
##    jnzflike   numlikenzf count
##    <fct>           <dbl> <int>
##  1 0                   0   622
##  2 1                   1   298
##  3 10                 10   134
##  4 2                   2   266
##  5 3                   3   227
##  6 4                   4   162
##  7 5                   5   544
##  8 6                   6   165
##  9 7                   7   138
## 10 8                   8   107
## 11 9                   9    81
## 12 Don't know         NA   224
## 13 <NA>               NA   133
\end{verbatim}

Converting the factor to a character first ensures that the numerical
values used in the labels of the levels of the categorical variable are
used.

Now that we cleaned up the data in a way that addresses the needs of the
research questions we want to explore, we are ready to continue with our
analysis.

\subsection{Appendix: List of fields in example
data}\label{appendix-list-of-fields-in-example-data}

\begin{longtable}[]{@{}lll@{}}
\toprule
\begin{minipage}[b]{0.14\columnwidth}\raggedright\strut
Variable\strut
\end{minipage} & \begin{minipage}[b]{0.70\columnwidth}\raggedright\strut
Question\strut
\end{minipage} & \begin{minipage}[b]{0.08\columnwidth}\raggedright\strut
DataType\strut
\end{minipage}\tabularnewline
\midrule
\endhead
\begin{minipage}[t]{0.14\columnwidth}\raggedright\strut
\texttt{jactlike}\strut
\end{minipage} & \begin{minipage}[t]{0.70\columnwidth}\raggedright\strut
A14: how much like Act\strut
\end{minipage} & \begin{minipage}[t]{0.08\columnwidth}\raggedright\strut
Factor\strut
\end{minipage}\tabularnewline
\begin{minipage}[t]{0.14\columnwidth}\raggedright\strut
\texttt{jactlr}\strut
\end{minipage} & \begin{minipage}[t]{0.70\columnwidth}\raggedright\strut
A18: Act on left-right scale\strut
\end{minipage} & \begin{minipage}[t]{0.08\columnwidth}\raggedright\strut
chr\strut
\end{minipage}\tabularnewline
\begin{minipage}[t]{0.14\columnwidth}\raggedright\strut
\texttt{jage}\strut
\end{minipage} & \begin{minipage}[t]{0.70\columnwidth}\raggedright\strut
Respondent's age in years\strut
\end{minipage} & \begin{minipage}[t]{0.08\columnwidth}\raggedright\strut
int\strut
\end{minipage}\tabularnewline
\begin{minipage}[t]{0.14\columnwidth}\raggedright\strut
\texttt{jblogel}\strut
\end{minipage} & \begin{minipage}[t]{0.70\columnwidth}\raggedright\strut
A6h: visit political blog for election\strut
\end{minipage} & \begin{minipage}[t]{0.08\columnwidth}\raggedright\strut
chr\strut
\end{minipage}\tabularnewline
\begin{minipage}[t]{0.14\columnwidth}\raggedright\strut
\texttt{jdiffvoting}\strut
\end{minipage} & \begin{minipage}[t]{0.70\columnwidth}\raggedright\strut
A13: does voting make any difference to what happens\strut
\end{minipage} & \begin{minipage}[t]{0.08\columnwidth}\raggedright\strut
chr\strut
\end{minipage}\tabularnewline
\begin{minipage}[t]{0.14\columnwidth}\raggedright\strut
\texttt{jdiscussp}\strut
\end{minipage} & \begin{minipage}[t]{0.70\columnwidth}\raggedright\strut
A11a: how often discussed politics with others\strut
\end{minipage} & \begin{minipage}[t]{0.08\columnwidth}\raggedright\strut
chr\strut
\end{minipage}\tabularnewline
\begin{minipage}[t]{0.14\columnwidth}\raggedright\strut
\texttt{Jelect}\strut
\end{minipage} & \begin{minipage}[t]{0.70\columnwidth}\raggedright\strut
Electorate\strut
\end{minipage} & \begin{minipage}[t]{0.08\columnwidth}\raggedright\strut
int\strut
\end{minipage}\tabularnewline
\begin{minipage}[t]{0.14\columnwidth}\raggedright\strut
\texttt{jelecvote}\strut
\end{minipage} & \begin{minipage}[t]{0.70\columnwidth}\raggedright\strut
C4: if cast electorate vote, for which party's candidate\strut
\end{minipage} & \begin{minipage}[t]{0.08\columnwidth}\raggedright\strut
chr\strut
\end{minipage}\tabularnewline
\begin{minipage}[t]{0.14\columnwidth}\raggedright\strut
\texttt{jethnicity\_a}\strut
\end{minipage} & \begin{minipage}[t]{0.70\columnwidth}\raggedright\strut
F19a: ethnicity - Asian\strut
\end{minipage} & \begin{minipage}[t]{0.08\columnwidth}\raggedright\strut
chr\strut
\end{minipage}\tabularnewline
\begin{minipage}[t]{0.14\columnwidth}\raggedright\strut
\texttt{jethnicity\_e}\strut
\end{minipage} & \begin{minipage}[t]{0.70\columnwidth}\raggedright\strut
F19a: ethnicity - NZ European\strut
\end{minipage} & \begin{minipage}[t]{0.08\columnwidth}\raggedright\strut
chr\strut
\end{minipage}\tabularnewline
\begin{minipage}[t]{0.14\columnwidth}\raggedright\strut
\texttt{jethnicity\_m}\strut
\end{minipage} & \begin{minipage}[t]{0.70\columnwidth}\raggedright\strut
F19a: ethnicity - NZ Maori\strut
\end{minipage} & \begin{minipage}[t]{0.08\columnwidth}\raggedright\strut
chr\strut
\end{minipage}\tabularnewline
\begin{minipage}[t]{0.14\columnwidth}\raggedright\strut
\texttt{jethnicity\_o}\strut
\end{minipage} & \begin{minipage}[t]{0.70\columnwidth}\raggedright\strut
F19a: ethnicity - Other\strut
\end{minipage} & \begin{minipage}[t]{0.08\columnwidth}\raggedright\strut
chr\strut
\end{minipage}\tabularnewline
\begin{minipage}[t]{0.14\columnwidth}\raggedright\strut
\texttt{jethnicity\_p}\strut
\end{minipage} & \begin{minipage}[t]{0.70\columnwidth}\raggedright\strut
F19a: ethnicity - Pacific\strut
\end{minipage} & \begin{minipage}[t]{0.08\columnwidth}\raggedright\strut
chr\strut
\end{minipage}\tabularnewline
\begin{minipage}[t]{0.14\columnwidth}\raggedright\strut
\texttt{jethnicityx}\strut
\end{minipage} & \begin{minipage}[t]{0.70\columnwidth}\raggedright\strut
F19ax: other ethnic group belonged to detail\strut
\end{minipage} & \begin{minipage}[t]{0.08\columnwidth}\raggedright\strut
chr\strut
\end{minipage}\tabularnewline
\begin{minipage}[t]{0.14\columnwidth}\raggedright\strut
\texttt{jethnicmost}\strut
\end{minipage} & \begin{minipage}[t]{0.70\columnwidth}\raggedright\strut
F19b: ethnic group identified with most\strut
\end{minipage} & \begin{minipage}[t]{0.08\columnwidth}\raggedright\strut
chr\strut
\end{minipage}\tabularnewline
\begin{minipage}[t]{0.14\columnwidth}\raggedright\strut
\texttt{jethnicmostx}\strut
\end{minipage} & \begin{minipage}[t]{0.70\columnwidth}\raggedright\strut
F19bx: other ethnic group identified with most\strut
\end{minipage} & \begin{minipage}[t]{0.08\columnwidth}\raggedright\strut
chr\strut
\end{minipage}\tabularnewline
\begin{minipage}[t]{0.14\columnwidth}\raggedright\strut
\texttt{jfirstpx}\strut
\end{minipage} & \begin{minipage}[t]{0.70\columnwidth}\raggedright\strut
C10x: on election day other party most wanted to be in government\strut
\end{minipage} & \begin{minipage}[t]{0.08\columnwidth}\raggedright\strut
chr\strut
\end{minipage}\tabularnewline
\begin{minipage}[t]{0.14\columnwidth}\raggedright\strut
\texttt{jgovpact}\strut
\end{minipage} & \begin{minipage}[t]{0.70\columnwidth}\raggedright\strut
C8: Act helped form the government after 2008 election\strut
\end{minipage} & \begin{minipage}[t]{0.08\columnwidth}\raggedright\strut
chr\strut
\end{minipage}\tabularnewline
\begin{minipage}[t]{0.14\columnwidth}\raggedright\strut
\texttt{jgovpdk}\strut
\end{minipage} & \begin{minipage}[t]{0.70\columnwidth}\raggedright\strut
C8: can't recall which parties formed the government after 2008
election\strut
\end{minipage} & \begin{minipage}[t]{0.08\columnwidth}\raggedright\strut
chr\strut
\end{minipage}\tabularnewline
\begin{minipage}[t]{0.14\columnwidth}\raggedright\strut
\texttt{jgovpgrn}\strut
\end{minipage} & \begin{minipage}[t]{0.70\columnwidth}\raggedright\strut
C8: Greens helped form the government after 2008 election\strut
\end{minipage} & \begin{minipage}[t]{0.08\columnwidth}\raggedright\strut
chr\strut
\end{minipage}\tabularnewline
\begin{minipage}[t]{0.14\columnwidth}\raggedright\strut
\texttt{jgovplab}\strut
\end{minipage} & \begin{minipage}[t]{0.70\columnwidth}\raggedright\strut
C8: Labour helped form the government after 2008 election\strut
\end{minipage} & \begin{minipage}[t]{0.08\columnwidth}\raggedright\strut
chr\strut
\end{minipage}\tabularnewline
\begin{minipage}[t]{0.14\columnwidth}\raggedright\strut
\texttt{jgovpmao}\strut
\end{minipage} & \begin{minipage}[t]{0.70\columnwidth}\raggedright\strut
C8: Maori Party helped form the government after 2008 election\strut
\end{minipage} & \begin{minipage}[t]{0.08\columnwidth}\raggedright\strut
chr\strut
\end{minipage}\tabularnewline
\begin{minipage}[t]{0.14\columnwidth}\raggedright\strut
\texttt{jgovpmnp}\strut
\end{minipage} & \begin{minipage}[t]{0.70\columnwidth}\raggedright\strut
C8: Mana Party helped form the government after 2008 election\strut
\end{minipage} & \begin{minipage}[t]{0.08\columnwidth}\raggedright\strut
chr\strut
\end{minipage}\tabularnewline
\begin{minipage}[t]{0.14\columnwidth}\raggedright\strut
\texttt{jgovpnat}\strut
\end{minipage} & \begin{minipage}[t]{0.70\columnwidth}\raggedright\strut
C8: National helped form the government after 2008 election\strut
\end{minipage} & \begin{minipage}[t]{0.08\columnwidth}\raggedright\strut
chr\strut
\end{minipage}\tabularnewline
\begin{minipage}[t]{0.14\columnwidth}\raggedright\strut
\texttt{jgovpnzf}\strut
\end{minipage} & \begin{minipage}[t]{0.70\columnwidth}\raggedright\strut
C8: NZ First helped form the government after 2008 election\strut
\end{minipage} & \begin{minipage}[t]{0.08\columnwidth}\raggedright\strut
chr\strut
\end{minipage}\tabularnewline
\begin{minipage}[t]{0.14\columnwidth}\raggedright\strut
\texttt{jgovunf}\strut
\end{minipage} & \begin{minipage}[t]{0.70\columnwidth}\raggedright\strut
C8: United Future helped form the government after 2008 election\strut
\end{minipage} & \begin{minipage}[t]{0.08\columnwidth}\raggedright\strut
chr\strut
\end{minipage}\tabularnewline
\begin{minipage}[t]{0.14\columnwidth}\raggedright\strut
\texttt{jgrnlike}\strut
\end{minipage} & \begin{minipage}[t]{0.70\columnwidth}\raggedright\strut
A14: how much like Greens\strut
\end{minipage} & \begin{minipage}[t]{0.08\columnwidth}\raggedright\strut
Factor\strut
\end{minipage}\tabularnewline
\begin{minipage}[t]{0.14\columnwidth}\raggedright\strut
\texttt{jgrnlr}\strut
\end{minipage} & \begin{minipage}[t]{0.70\columnwidth}\raggedright\strut
A18: Greens on left-right scale\strut
\end{minipage} & \begin{minipage}[t]{0.08\columnwidth}\raggedright\strut
chr\strut
\end{minipage}\tabularnewline
\begin{minipage}[t]{0.14\columnwidth}\raggedright\strut
\texttt{jhhadults}\strut
\end{minipage} & \begin{minipage}[t]{0.70\columnwidth}\raggedright\strut
F23a: number of adults in household\strut
\end{minipage} & \begin{minipage}[t]{0.08\columnwidth}\raggedright\strut
int\strut
\end{minipage}\tabularnewline
\begin{minipage}[t]{0.14\columnwidth}\raggedright\strut
\texttt{jhhchn}\strut
\end{minipage} & \begin{minipage}[t]{0.70\columnwidth}\raggedright\strut
F23b: number of children in household\strut
\end{minipage} & \begin{minipage}[t]{0.08\columnwidth}\raggedright\strut
int\strut
\end{minipage}\tabularnewline
\begin{minipage}[t]{0.14\columnwidth}\raggedright\strut
\texttt{jhhincome}\strut
\end{minipage} & \begin{minipage}[t]{0.70\columnwidth}\raggedright\strut
F22: household income between 1 April 2010 and 31 March 2011\strut
\end{minipage} & \begin{minipage}[t]{0.08\columnwidth}\raggedright\strut
chr\strut
\end{minipage}\tabularnewline
\begin{minipage}[t]{0.14\columnwidth}\raggedright\strut
\texttt{jhqual}\strut
\end{minipage} & \begin{minipage}[t]{0.70\columnwidth}\raggedright\strut
F8: highest formal educational qualification\strut
\end{minipage} & \begin{minipage}[t]{0.08\columnwidth}\raggedright\strut
chr\strut
\end{minipage}\tabularnewline
\begin{minipage}[t]{0.14\columnwidth}\raggedright\strut
\texttt{jlablike}\strut
\end{minipage} & \begin{minipage}[t]{0.70\columnwidth}\raggedright\strut
A14: how much like Labour\strut
\end{minipage} & \begin{minipage}[t]{0.08\columnwidth}\raggedright\strut
Factor\strut
\end{minipage}\tabularnewline
\begin{minipage}[t]{0.14\columnwidth}\raggedright\strut
\texttt{jlablr}\strut
\end{minipage} & \begin{minipage}[t]{0.70\columnwidth}\raggedright\strut
A18: Labour on left-right scale\strut
\end{minipage} & \begin{minipage}[t]{0.08\columnwidth}\raggedright\strut
chr\strut
\end{minipage}\tabularnewline
\begin{minipage}[t]{0.14\columnwidth}\raggedright\strut
\texttt{jlanguage}\strut
\end{minipage} & \begin{minipage}[t]{0.70\columnwidth}\raggedright\strut
F3: main language spoken at your home\strut
\end{minipage} & \begin{minipage}[t]{0.08\columnwidth}\raggedright\strut
chr\strut
\end{minipage}\tabularnewline
\begin{minipage}[t]{0.14\columnwidth}\raggedright\strut
\texttt{jlanguagex}\strut
\end{minipage} & \begin{minipage}[t]{0.70\columnwidth}\raggedright\strut
F3x: other main language spoken\strut
\end{minipage} & \begin{minipage}[t]{0.08\columnwidth}\raggedright\strut
chr\strut
\end{minipage}\tabularnewline
\begin{minipage}[t]{0.14\columnwidth}\raggedright\strut
\texttt{jmaolike}\strut
\end{minipage} & \begin{minipage}[t]{0.70\columnwidth}\raggedright\strut
A14: how much like Maori Party\strut
\end{minipage} & \begin{minipage}[t]{0.08\columnwidth}\raggedright\strut
Factor\strut
\end{minipage}\tabularnewline
\begin{minipage}[t]{0.14\columnwidth}\raggedright\strut
\texttt{jmaolr}\strut
\end{minipage} & \begin{minipage}[t]{0.70\columnwidth}\raggedright\strut
A18: Maori Party on left-right scale\strut
\end{minipage} & \begin{minipage}[t]{0.08\columnwidth}\raggedright\strut
chr\strut
\end{minipage}\tabularnewline
\begin{minipage}[t]{0.14\columnwidth}\raggedright\strut
\texttt{jmarital}\strut
\end{minipage} & \begin{minipage}[t]{0.70\columnwidth}\raggedright\strut
F24: marital status\strut
\end{minipage} & \begin{minipage}[t]{0.08\columnwidth}\raggedright\strut
chr\strut
\end{minipage}\tabularnewline
\begin{minipage}[t]{0.14\columnwidth}\raggedright\strut
\texttt{jmnplike}\strut
\end{minipage} & \begin{minipage}[t]{0.70\columnwidth}\raggedright\strut
A14: how much like Mana Party\strut
\end{minipage} & \begin{minipage}[t]{0.08\columnwidth}\raggedright\strut
Factor\strut
\end{minipage}\tabularnewline
\begin{minipage}[t]{0.14\columnwidth}\raggedright\strut
\texttt{jmnplr}\strut
\end{minipage} & \begin{minipage}[t]{0.70\columnwidth}\raggedright\strut
A18: Mana Party on left-right scale\strut
\end{minipage} & \begin{minipage}[t]{0.08\columnwidth}\raggedright\strut
chr\strut
\end{minipage}\tabularnewline
\begin{minipage}[t]{0.14\columnwidth}\raggedright\strut
\texttt{jmostlike}\strut
\end{minipage} & \begin{minipage}[t]{0.70\columnwidth}\raggedright\strut
A15: on election day which party liked most\strut
\end{minipage} & \begin{minipage}[t]{0.08\columnwidth}\raggedright\strut
chr\strut
\end{minipage}\tabularnewline
\begin{minipage}[t]{0.14\columnwidth}\raggedright\strut
\texttt{jmostlikex}\strut
\end{minipage} & \begin{minipage}[t]{0.70\columnwidth}\raggedright\strut
A15x: other party liked most\strut
\end{minipage} & \begin{minipage}[t]{0.08\columnwidth}\raggedright\strut
chr\strut
\end{minipage}\tabularnewline
\begin{minipage}[t]{0.14\columnwidth}\raggedright\strut
\texttt{jnatlike}\strut
\end{minipage} & \begin{minipage}[t]{0.70\columnwidth}\raggedright\strut
A14: how much like National\strut
\end{minipage} & \begin{minipage}[t]{0.08\columnwidth}\raggedright\strut
Factor\strut
\end{minipage}\tabularnewline
\begin{minipage}[t]{0.14\columnwidth}\raggedright\strut
\texttt{jnatlr}\strut
\end{minipage} & \begin{minipage}[t]{0.70\columnwidth}\raggedright\strut
A18: National on left-right scale\strut
\end{minipage} & \begin{minipage}[t]{0.08\columnwidth}\raggedright\strut
chr\strut
\end{minipage}\tabularnewline
\begin{minipage}[t]{0.14\columnwidth}\raggedright\strut
\texttt{jnatradio}\strut
\end{minipage} & \begin{minipage}[t]{0.70\columnwidth}\raggedright\strut
A10d: how often followed election news on Radio New Zealand:
National\strut
\end{minipage} & \begin{minipage}[t]{0.08\columnwidth}\raggedright\strut
chr\strut
\end{minipage}\tabularnewline
\begin{minipage}[t]{0.14\columnwidth}\raggedright\strut
\texttt{jnevervoteact}\strut
\end{minipage} & \begin{minipage}[t]{0.70\columnwidth}\raggedright\strut
C16: would never vote for Act\strut
\end{minipage} & \begin{minipage}[t]{0.08\columnwidth}\raggedright\strut
chr\strut
\end{minipage}\tabularnewline
\begin{minipage}[t]{0.14\columnwidth}\raggedright\strut
\texttt{jnevervotegrn}\strut
\end{minipage} & \begin{minipage}[t]{0.70\columnwidth}\raggedright\strut
C16: would never vote for Greens\strut
\end{minipage} & \begin{minipage}[t]{0.08\columnwidth}\raggedright\strut
chr\strut
\end{minipage}\tabularnewline
\begin{minipage}[t]{0.14\columnwidth}\raggedright\strut
\texttt{jnevervotelab}\strut
\end{minipage} & \begin{minipage}[t]{0.70\columnwidth}\raggedright\strut
C16: would never vote for Labour\strut
\end{minipage} & \begin{minipage}[t]{0.08\columnwidth}\raggedright\strut
chr\strut
\end{minipage}\tabularnewline
\begin{minipage}[t]{0.14\columnwidth}\raggedright\strut
\texttt{jnevervotemao}\strut
\end{minipage} & \begin{minipage}[t]{0.70\columnwidth}\raggedright\strut
C16: would never vote for Maori Party\strut
\end{minipage} & \begin{minipage}[t]{0.08\columnwidth}\raggedright\strut
chr\strut
\end{minipage}\tabularnewline
\begin{minipage}[t]{0.14\columnwidth}\raggedright\strut
\texttt{jnevervotemnp}\strut
\end{minipage} & \begin{minipage}[t]{0.70\columnwidth}\raggedright\strut
C16: would never vote for Mana Party\strut
\end{minipage} & \begin{minipage}[t]{0.08\columnwidth}\raggedright\strut
chr\strut
\end{minipage}\tabularnewline
\begin{minipage}[t]{0.14\columnwidth}\raggedright\strut
\texttt{jnevervotenat}\strut
\end{minipage} & \begin{minipage}[t]{0.70\columnwidth}\raggedright\strut
C16: would never vote for National\strut
\end{minipage} & \begin{minipage}[t]{0.08\columnwidth}\raggedright\strut
chr\strut
\end{minipage}\tabularnewline
\begin{minipage}[t]{0.14\columnwidth}\raggedright\strut
\texttt{jnevervoteno}\strut
\end{minipage} & \begin{minipage}[t]{0.70\columnwidth}\raggedright\strut
C16: no party for which you would never vote\strut
\end{minipage} & \begin{minipage}[t]{0.08\columnwidth}\raggedright\strut
chr\strut
\end{minipage}\tabularnewline
\begin{minipage}[t]{0.14\columnwidth}\raggedright\strut
\texttt{jnevervotenzf}\strut
\end{minipage} & \begin{minipage}[t]{0.70\columnwidth}\raggedright\strut
C16: would never vote for NZ First\strut
\end{minipage} & \begin{minipage}[t]{0.08\columnwidth}\raggedright\strut
chr\strut
\end{minipage}\tabularnewline
\begin{minipage}[t]{0.14\columnwidth}\raggedright\strut
\texttt{jnevervoteoth}\strut
\end{minipage} & \begin{minipage}[t]{0.70\columnwidth}\raggedright\strut
C16: would never vote for another party\strut
\end{minipage} & \begin{minipage}[t]{0.08\columnwidth}\raggedright\strut
chr\strut
\end{minipage}\tabularnewline
\begin{minipage}[t]{0.14\columnwidth}\raggedright\strut
\texttt{jnevervoteothx}\strut
\end{minipage} & \begin{minipage}[t]{0.70\columnwidth}\raggedright\strut
C16: other party for for which you would never vote\strut
\end{minipage} & \begin{minipage}[t]{0.08\columnwidth}\raggedright\strut
chr\strut
\end{minipage}\tabularnewline
\begin{minipage}[t]{0.14\columnwidth}\raggedright\strut
\texttt{jnevervoteunf}\strut
\end{minipage} & \begin{minipage}[t]{0.70\columnwidth}\raggedright\strut
C16: would never vote for United Future\strut
\end{minipage} & \begin{minipage}[t]{0.08\columnwidth}\raggedright\strut
chr\strut
\end{minipage}\tabularnewline
\begin{minipage}[t]{0.14\columnwidth}\raggedright\strut
\texttt{jnewspaper}\strut
\end{minipage} & \begin{minipage}[t]{0.70\columnwidth}\raggedright\strut
A10c: how often followed election news in newspaper\strut
\end{minipage} & \begin{minipage}[t]{0.08\columnwidth}\raggedright\strut
chr\strut
\end{minipage}\tabularnewline
\begin{minipage}[t]{0.14\columnwidth}\raggedright\strut
\texttt{jnzflike}\strut
\end{minipage} & \begin{minipage}[t]{0.70\columnwidth}\raggedright\strut
A14: how much like NZ First\strut
\end{minipage} & \begin{minipage}[t]{0.08\columnwidth}\raggedright\strut
Factor\strut
\end{minipage}\tabularnewline
\begin{minipage}[t]{0.14\columnwidth}\raggedright\strut
\texttt{jnzflr}\strut
\end{minipage} & \begin{minipage}[t]{0.70\columnwidth}\raggedright\strut
A18: NZ First on left-right scale\strut
\end{minipage} & \begin{minipage}[t]{0.08\columnwidth}\raggedright\strut
chr\strut
\end{minipage}\tabularnewline
\begin{minipage}[t]{0.14\columnwidth}\raggedright\strut
\texttt{jpartyvote}\strut
\end{minipage} & \begin{minipage}[t]{0.70\columnwidth}\raggedright\strut
C3: if cast party vote, for which party\strut
\end{minipage} & \begin{minipage}[t]{0.08\columnwidth}\raggedright\strut
chr\strut
\end{minipage}\tabularnewline
\begin{minipage}[t]{0.14\columnwidth}\raggedright\strut
\texttt{jpcmoney}\strut
\end{minipage} & \begin{minipage}[t]{0.70\columnwidth}\raggedright\strut
A11d: how often contributed money to a party or candidate\strut
\end{minipage} & \begin{minipage}[t]{0.08\columnwidth}\raggedright\strut
chr\strut
\end{minipage}\tabularnewline
\begin{minipage}[t]{0.14\columnwidth}\raggedright\strut
\texttt{jpcposter}\strut
\end{minipage} & \begin{minipage}[t]{0.70\columnwidth}\raggedright\strut
A11e: how often put up party or candidate posters\strut
\end{minipage} & \begin{minipage}[t]{0.08\columnwidth}\raggedright\strut
chr\strut
\end{minipage}\tabularnewline
\begin{minipage}[t]{0.14\columnwidth}\raggedright\strut
\texttt{jpersuade}\strut
\end{minipage} & \begin{minipage}[t]{0.70\columnwidth}\raggedright\strut
A11c: how often talk to anyone to persuade them how to vote\strut
\end{minipage} & \begin{minipage}[t]{0.08\columnwidth}\raggedright\strut
chr\strut
\end{minipage}\tabularnewline
\begin{minipage}[t]{0.14\columnwidth}\raggedright\strut
\texttt{jrallies}\strut
\end{minipage} & \begin{minipage}[t]{0.70\columnwidth}\raggedright\strut
A11b: how often attended political meetings or rallies\strut
\end{minipage} & \begin{minipage}[t]{0.08\columnwidth}\raggedright\strut
chr\strut
\end{minipage}\tabularnewline
\begin{minipage}[t]{0.14\columnwidth}\raggedright\strut
\texttt{jrelang}\strut
\end{minipage} & \begin{minipage}[t]{0.70\columnwidth}\raggedright\strut
F17: anglican\strut
\end{minipage} & \begin{minipage}[t]{0.08\columnwidth}\raggedright\strut
chr\strut
\end{minipage}\tabularnewline
\begin{minipage}[t]{0.14\columnwidth}\raggedright\strut
\texttt{jrelbap}\strut
\end{minipage} & \begin{minipage}[t]{0.70\columnwidth}\raggedright\strut
F17: baptist\strut
\end{minipage} & \begin{minipage}[t]{0.08\columnwidth}\raggedright\strut
chr\strut
\end{minipage}\tabularnewline
\begin{minipage}[t]{0.14\columnwidth}\raggedright\strut
\texttt{jrelcath}\strut
\end{minipage} & \begin{minipage}[t]{0.70\columnwidth}\raggedright\strut
F17: catholic\strut
\end{minipage} & \begin{minipage}[t]{0.08\columnwidth}\raggedright\strut
chr\strut
\end{minipage}\tabularnewline
\begin{minipage}[t]{0.14\columnwidth}\raggedright\strut
\texttt{jrelfun}\strut
\end{minipage} & \begin{minipage}[t]{0.70\columnwidth}\raggedright\strut
F17: independent-fundamentalist-pentecostal church\strut
\end{minipage} & \begin{minipage}[t]{0.08\columnwidth}\raggedright\strut
chr\strut
\end{minipage}\tabularnewline
\begin{minipage}[t]{0.14\columnwidth}\raggedright\strut
\texttt{jreligionx}\strut
\end{minipage} & \begin{minipage}[t]{0.70\columnwidth}\raggedright\strut
F17x: other religion detail\strut
\end{minipage} & \begin{minipage}[t]{0.08\columnwidth}\raggedright\strut
chr\strut
\end{minipage}\tabularnewline
\begin{minipage}[t]{0.14\columnwidth}\raggedright\strut
\texttt{jreligiousity}\strut
\end{minipage} & \begin{minipage}[t]{0.70\columnwidth}\raggedright\strut
F18: how religious are you\strut
\end{minipage} & \begin{minipage}[t]{0.08\columnwidth}\raggedright\strut
chr\strut
\end{minipage}\tabularnewline
\begin{minipage}[t]{0.14\columnwidth}\raggedright\strut
\texttt{jrellat}\strut
\end{minipage} & \begin{minipage}[t]{0.70\columnwidth}\raggedright\strut
F17: latter day saints\strut
\end{minipage} & \begin{minipage}[t]{0.08\columnwidth}\raggedright\strut
chr\strut
\end{minipage}\tabularnewline
\begin{minipage}[t]{0.14\columnwidth}\raggedright\strut
\texttt{jrelmeth}\strut
\end{minipage} & \begin{minipage}[t]{0.70\columnwidth}\raggedright\strut
F17: methodist\strut
\end{minipage} & \begin{minipage}[t]{0.08\columnwidth}\raggedright\strut
chr\strut
\end{minipage}\tabularnewline
\begin{minipage}[t]{0.14\columnwidth}\raggedright\strut
\texttt{jrelnonc}\strut
\end{minipage} & \begin{minipage}[t]{0.70\columnwidth}\raggedright\strut
F17: non-Christian\strut
\end{minipage} & \begin{minipage}[t]{0.08\columnwidth}\raggedright\strut
chr\strut
\end{minipage}\tabularnewline
\begin{minipage}[t]{0.14\columnwidth}\raggedright\strut
\texttt{jrelnone}\strut
\end{minipage} & \begin{minipage}[t]{0.70\columnwidth}\raggedright\strut
F17: no religion\strut
\end{minipage} & \begin{minipage}[t]{0.08\columnwidth}\raggedright\strut
chr\strut
\end{minipage}\tabularnewline
\begin{minipage}[t]{0.14\columnwidth}\raggedright\strut
\texttt{jrelothc}\strut
\end{minipage} & \begin{minipage}[t]{0.70\columnwidth}\raggedright\strut
F17: other Christian\strut
\end{minipage} & \begin{minipage}[t]{0.08\columnwidth}\raggedright\strut
chr\strut
\end{minipage}\tabularnewline
\begin{minipage}[t]{0.14\columnwidth}\raggedright\strut
\texttt{jrelpres}\strut
\end{minipage} & \begin{minipage}[t]{0.70\columnwidth}\raggedright\strut
F17: presbyterian\strut
\end{minipage} & \begin{minipage}[t]{0.08\columnwidth}\raggedright\strut
chr\strut
\end{minipage}\tabularnewline
\begin{minipage}[t]{0.14\columnwidth}\raggedright\strut
\texttt{jrelrat}\strut
\end{minipage} & \begin{minipage}[t]{0.70\columnwidth}\raggedright\strut
F17: ratana\strut
\end{minipage} & \begin{minipage}[t]{0.08\columnwidth}\raggedright\strut
chr\strut
\end{minipage}\tabularnewline
\begin{minipage}[t]{0.14\columnwidth}\raggedright\strut
\texttt{jrelservices}\strut
\end{minipage} & \begin{minipage}[t]{0.70\columnwidth}\raggedright\strut
F16: apart from weddings, funerals, baptisms, how often do you attend
religious services\strut
\end{minipage} & \begin{minipage}[t]{0.08\columnwidth}\raggedright\strut
chr\strut
\end{minipage}\tabularnewline
\begin{minipage}[t]{0.14\columnwidth}\raggedright\strut
\texttt{jrepublic}\strut
\end{minipage} & \begin{minipage}[t]{0.70\columnwidth}\raggedright\strut
B1: should NZ become a republic or retain Queen as head of state\strut
\end{minipage} & \begin{minipage}[t]{0.08\columnwidth}\raggedright\strut
chr\strut
\end{minipage}\tabularnewline
\begin{minipage}[t]{0.14\columnwidth}\raggedright\strut
\texttt{jrollsex}\strut
\end{minipage} & \begin{minipage}[t]{0.70\columnwidth}\raggedright\strut
Respondent's gender from electoral roll\strut
\end{minipage} & \begin{minipage}[t]{0.08\columnwidth}\raggedright\strut
chr\strut
\end{minipage}\tabularnewline
\begin{minipage}[t]{0.14\columnwidth}\raggedright\strut
\texttt{jsecondp}\strut
\end{minipage} & \begin{minipage}[t]{0.70\columnwidth}\raggedright\strut
C11: on election day which party overall was you second choice to be in
government\strut
\end{minipage} & \begin{minipage}[t]{0.08\columnwidth}\raggedright\strut
chr\strut
\end{minipage}\tabularnewline
\begin{minipage}[t]{0.14\columnwidth}\raggedright\strut
\texttt{jslflr}\strut
\end{minipage} & \begin{minipage}[t]{0.70\columnwidth}\raggedright\strut
A19: yourself on left-right scale\strut
\end{minipage} & \begin{minipage}[t]{0.08\columnwidth}\raggedright\strut
chr\strut
\end{minipage}\tabularnewline
\begin{minipage}[t]{0.14\columnwidth}\raggedright\strut
\texttt{jspbusind}\strut
\end{minipage} & \begin{minipage}[t]{0.70\columnwidth}\raggedright\strut
B3f: should there be more or less public spending on business and
industry\strut
\end{minipage} & \begin{minipage}[t]{0.08\columnwidth}\raggedright\strut
chr\strut
\end{minipage}\tabularnewline
\begin{minipage}[t]{0.14\columnwidth}\raggedright\strut
\texttt{jspdefence}\strut
\end{minipage} & \begin{minipage}[t]{0.70\columnwidth}\raggedright\strut
B3d: should there be more or less public spending on defence\strut
\end{minipage} & \begin{minipage}[t]{0.08\columnwidth}\raggedright\strut
chr\strut
\end{minipage}\tabularnewline
\begin{minipage}[t]{0.14\columnwidth}\raggedright\strut
\texttt{jspedu}\strut
\end{minipage} & \begin{minipage}[t]{0.70\columnwidth}\raggedright\strut
B3b: should there be more or less public spending on education\strut
\end{minipage} & \begin{minipage}[t]{0.08\columnwidth}\raggedright\strut
chr\strut
\end{minipage}\tabularnewline
\begin{minipage}[t]{0.14\columnwidth}\raggedright\strut
\texttt{jspenviro}\strut
\end{minipage} & \begin{minipage}[t]{0.70\columnwidth}\raggedright\strut
B3i: should there be more or less public spending on the
environment\strut
\end{minipage} & \begin{minipage}[t]{0.08\columnwidth}\raggedright\strut
chr\strut
\end{minipage}\tabularnewline
\begin{minipage}[t]{0.14\columnwidth}\raggedright\strut
\texttt{jsphealth}\strut
\end{minipage} & \begin{minipage}[t]{0.70\columnwidth}\raggedright\strut
B3a: should there be more or less public spending on health\strut
\end{minipage} & \begin{minipage}[t]{0.08\columnwidth}\raggedright\strut
chr\strut
\end{minipage}\tabularnewline
\begin{minipage}[t]{0.14\columnwidth}\raggedright\strut
\texttt{jsppolice}\strut
\end{minipage} & \begin{minipage}[t]{0.70\columnwidth}\raggedright\strut
B3g: should there be more or less public spending on police and law
enforcement\strut
\end{minipage} & \begin{minipage}[t]{0.08\columnwidth}\raggedright\strut
chr\strut
\end{minipage}\tabularnewline
\begin{minipage}[t]{0.14\columnwidth}\raggedright\strut
\texttt{jspsuper}\strut
\end{minipage} & \begin{minipage}[t]{0.70\columnwidth}\raggedright\strut
B3e: should there be more or less public spending on
superannuation\strut
\end{minipage} & \begin{minipage}[t]{0.08\columnwidth}\raggedright\strut
chr\strut
\end{minipage}\tabularnewline
\begin{minipage}[t]{0.14\columnwidth}\raggedright\strut
\texttt{jspunemp}\strut
\end{minipage} & \begin{minipage}[t]{0.70\columnwidth}\raggedright\strut
B3c: should there be more or less public spending on unemployment
benefits\strut
\end{minipage} & \begin{minipage}[t]{0.08\columnwidth}\raggedright\strut
chr\strut
\end{minipage}\tabularnewline
\begin{minipage}[t]{0.14\columnwidth}\raggedright\strut
\texttt{jspwelfare}\strut
\end{minipage} & \begin{minipage}[t]{0.70\columnwidth}\raggedright\strut
B3h: should there be more or less public spending on welfare
benefits\strut
\end{minipage} & \begin{minipage}[t]{0.08\columnwidth}\raggedright\strut
chr\strut
\end{minipage}\tabularnewline
\begin{minipage}[t]{0.14\columnwidth}\raggedright\strut
\texttt{jtalkback}\strut
\end{minipage} & \begin{minipage}[t]{0.70\columnwidth}\raggedright\strut
A10e: how often followed election news on talkback radio\strut
\end{minipage} & \begin{minipage}[t]{0.08\columnwidth}\raggedright\strut
chr\strut
\end{minipage}\tabularnewline
\begin{minipage}[t]{0.14\columnwidth}\raggedright\strut
\texttt{junflike}\strut
\end{minipage} & \begin{minipage}[t]{0.70\columnwidth}\raggedright\strut
A14: how much like United Future\strut
\end{minipage} & \begin{minipage}[t]{0.08\columnwidth}\raggedright\strut
Factor\strut
\end{minipage}\tabularnewline
\begin{minipage}[t]{0.14\columnwidth}\raggedright\strut
\texttt{junflr}\strut
\end{minipage} & \begin{minipage}[t]{0.70\columnwidth}\raggedright\strut
A18: United Future on left-right scale\strut
\end{minipage} & \begin{minipage}[t]{0.08\columnwidth}\raggedright\strut
chr\strut
\end{minipage}\tabularnewline
\begin{minipage}[t]{0.14\columnwidth}\raggedright\strut
\texttt{jwkdis}\strut
\end{minipage} & \begin{minipage}[t]{0.70\columnwidth}\raggedright\strut
F9: disabled, unable to work\strut
\end{minipage} & \begin{minipage}[t]{0.08\columnwidth}\raggedright\strut
chr\strut
\end{minipage}\tabularnewline
\begin{minipage}[t]{0.14\columnwidth}\raggedright\strut
\texttt{jwkft}\strut
\end{minipage} & \begin{minipage}[t]{0.70\columnwidth}\raggedright\strut
F9: working full-time for pay or other income\strut
\end{minipage} & \begin{minipage}[t]{0.08\columnwidth}\raggedright\strut
chr\strut
\end{minipage}\tabularnewline
\begin{minipage}[t]{0.14\columnwidth}\raggedright\strut
\texttt{jwkpt}\strut
\end{minipage} & \begin{minipage}[t]{0.70\columnwidth}\raggedright\strut
F9: working part-time for pay or other income\strut
\end{minipage} & \begin{minipage}[t]{0.08\columnwidth}\raggedright\strut
chr\strut
\end{minipage}\tabularnewline
\begin{minipage}[t]{0.14\columnwidth}\raggedright\strut
\texttt{jwkret}\strut
\end{minipage} & \begin{minipage}[t]{0.70\columnwidth}\raggedright\strut
F9: retired\strut
\end{minipage} & \begin{minipage}[t]{0.08\columnwidth}\raggedright\strut
chr\strut
\end{minipage}\tabularnewline
\begin{minipage}[t]{0.14\columnwidth}\raggedright\strut
\texttt{jwksch}\strut
\end{minipage} & \begin{minipage}[t]{0.70\columnwidth}\raggedright\strut
F9: at school, university, or other educational institution\strut
\end{minipage} & \begin{minipage}[t]{0.08\columnwidth}\raggedright\strut
chr\strut
\end{minipage}\tabularnewline
\begin{minipage}[t]{0.14\columnwidth}\raggedright\strut
\texttt{jwkun}\strut
\end{minipage} & \begin{minipage}[t]{0.70\columnwidth}\raggedright\strut
F9: unemployed, laid off, looking for work\strut
\end{minipage} & \begin{minipage}[t]{0.08\columnwidth}\raggedright\strut
chr\strut
\end{minipage}\tabularnewline
\begin{minipage}[t]{0.14\columnwidth}\raggedright\strut
\texttt{jwkunpi}\strut
\end{minipage} & \begin{minipage}[t]{0.70\columnwidth}\raggedright\strut
F9: working unpaid within the home\strut
\end{minipage} & \begin{minipage}[t]{0.08\columnwidth}\raggedright\strut
chr\strut
\end{minipage}\tabularnewline
\begin{minipage}[t]{0.14\columnwidth}\raggedright\strut
\texttt{jwkunpo}\strut
\end{minipage} & \begin{minipage}[t]{0.70\columnwidth}\raggedright\strut
F9: working unpaid outside the home\strut
\end{minipage} & \begin{minipage}[t]{0.08\columnwidth}\raggedright\strut
chr\strut
\end{minipage}\tabularnewline
\begin{minipage}[t]{0.14\columnwidth}\raggedright\strut
\texttt{njelecvote}\strut
\end{minipage} & \begin{minipage}[t]{0.70\columnwidth}\raggedright\strut
Electorate Vote with nonvote\strut
\end{minipage} & \begin{minipage}[t]{0.08\columnwidth}\raggedright\strut
chr\strut
\end{minipage}\tabularnewline
\begin{minipage}[t]{0.14\columnwidth}\raggedright\strut
\texttt{njptyvote}\strut
\end{minipage} & \begin{minipage}[t]{0.70\columnwidth}\raggedright\strut
Party Vote with nonvote\strut
\end{minipage} & \begin{minipage}[t]{0.08\columnwidth}\raggedright\strut
chr\strut
\end{minipage}\tabularnewline
\begin{minipage}[t]{0.14\columnwidth}\raggedright\strut
\texttt{r\_jind}\strut
\end{minipage} & \begin{minipage}[t]{0.70\columnwidth}\raggedright\strut
Respondent Industry Codes\strut
\end{minipage} & \begin{minipage}[t]{0.08\columnwidth}\raggedright\strut
chr\strut
\end{minipage}\tabularnewline
\begin{minipage}[t]{0.14\columnwidth}\raggedright\strut
\texttt{\_singlefav}\strut
\end{minipage} & \begin{minipage}[t]{0.70\columnwidth}\raggedright\strut
Caluclated Variable of most liked of major parties Question A14\strut
\end{minipage} & \begin{minipage}[t]{0.08\columnwidth}\raggedright\strut
chr\strut
\end{minipage}\tabularnewline
\bottomrule
\end{longtable}


\end{document}
